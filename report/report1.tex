\documentclass[a4paper,11pt]{jsarticle}

% 数式
\usepackage{amsmath,amsfonts,amssymb}
\usepackage{mathtools}
\usepackage{bm}

% 図・色・レイアウト系
\usepackage[x11names]{xcolor}         % まず xcolor を先に読み込む(colorより推奨)
\usepackage[dvipdfmx]{graphicx}       % graphicx は後に
\usepackage{here}                     % [H]で図の位置制御
\usepackage{type1cm} 
\usepackage{diagbox}
\usepackage[format=plain,justification=centering]{caption}
\usepackage{makecell} % プリアンブルに追加
\usepackage{array} % プリアンブルに追加

\usepackage{chngcntr}
\counterwithin{figure}{section}


% フレームや枠など
\usepackage{ascmac}
\usepackage{fancybox}

\usepackage{multirow}
\usepackage{booktabs}

% リストやコード
\usepackage{listings}
\usepackage{lmodern}
\usepackage{inconsolata}
\usepackage{array}

% ハイパーリンクと目次強化
\usepackage{hyperref}

% 箇条書きのカスタマイズ
\usepackage{enumerate}

\usepackage{algorithm}
\usepackage{algpseudocode}


% \LaTeX コマンド定義
\makeatletter
\providecommand{\LaTeX}{%
  L\kern-.36em\raise.3ex\hbox{\scshape a}\kern-.15em%
  T\kern-.1667em\lower.7ex\hbox{E}\kern-.125emX}
\makeatother

% listings 設定
\lstset{
  basicstyle=\ttfamily,
  identifierstyle=\small,
  commentstyle=\small\itshape,
  keywordstyle=\small\bfseries,
  ndkeywordstyle=\small,
  stringstyle=\small\ttfamily,
  frame={tb},
  breaklines=true,
  columns=[l]{fullflexible},
  numbers=none,
  xrightmargin=0zw,
  xleftmargin=3zw,
  numberstyle=\scriptsize,
  stepnumber=1,
  numbersep=1zw,
  lineskip=-0.5ex,
  literate={<}{{\textless}}1 {>}{{\textgreater}}1
}
\renewcommand{\thefigure}{\thesection-\arabic{figure}}
\newcommand{\grayhrulefill}{\textcolor{gray!50}{\rule[0.5ex]{\dimexpr\linewidth-1pt}{0.4pt}}}

% \subtitle コマンド定義と \maketitle の再定義
\makeatletter
\newcommand{\subtitle}[1]{\gdef\@subtitle{#1}}
\let\@subtitle\@empty
\renewcommand{\maketitle}{%
  \begin{center}
    {\LARGE \@title \par}
    \vskip 0.5em
    {\large \@subtitle \par}
    \vskip 1em
    {\large \@author \par}
    \vskip 1em
    {\normalsize \@date \par}
  \end{center}
  \vskip 2em
}
\makeatother

\begin{document}
\flushbottom

\title{総合演習課題0:小課題}
\subtitle{}
\author{学籍番号:2355078T \\ 氏名:小島魁人}
\date{\today}
\maketitle

\section{問0-1}




\section{問0-2}

以下に示す(1)式を変形し、講義スライドp.11の(2)式になることを示す。

\begin{equation}
  \begin{aligned}
    \frac{\partial v_x}{\partial t} + \left(v_x \frac{\partial}{\partial x} + v_y \frac{\partial}{\partial y}\right) v_x &= -\frac{1}{\rho} \frac{\partial p}{\partial x} + \nu \left(\frac{\partial^2}{\partial x^2} + \frac{\partial^2}{\partial y^2}\right) v_x \quad \cdots\text{①} \\
    \frac{\partial v_x}{\partial t} + \left(v_x \frac{\partial}{\partial x} + v_y \frac{\partial}{\partial y}\right) v_y &= -\frac{1}{\rho} \frac{\partial p}{\partial y} + \nu \left(\frac{\partial^2}{\partial x^2} + \frac{\partial^2}{\partial y^2}\right) v_y \quad \cdots\text{②}\\
    \frac{\partial v_x}{\partial x} + \frac{\partial v_y}{\partial y} &= 0 \quad \cdots\text{③}
  \end{aligned}
\end{equation}

\textbf{ⅰ}\(\qquad \frac{\partial}{\partial x}\text{②} - \frac{\partial}{\partial y}\text{①}\)とする。

\begin{equation*}
  \frac{\partial}{\partial t}\left(\frac{\partial v_y}{\partial x} - \frac{\partial v_x}{\partial y}\right) + \left(v_x \frac{\partial}{\partial x} + v_y \frac{\partial}{\partial y}\right)\left(\frac{\partial v_y}{\partial x} - \frac{\partial v_x}{\partial y}\right) = \nu\left(\frac{\partial^2}{\partial x^2} + \frac{\partial^2}{\partial y^2}\right)\left(\frac{\partial v_y}{\partial x} - \frac{\partial v_x}{\partial y}\right)
\end{equation*}

ここで、\(\omega = \frac{\partial v_y}{\partial x} - \frac{\partial v_x}{\partial y} \quad \cdots \text{④}\)とすると、

\begin{align*}
  -\frac{\partial \omega}{\partial t} - \left(v_x \frac{\partial}{\partial x} + v_y \frac{\partial}{\partial y}\right)\omega &= - \nu \left(\frac{\partial^2}{\partial x^2} + \frac{\partial^2}{\partial y^2}\right)\omega \\ 
  \frac{\partial \omega}{\partial t} + \left(v_x \frac{\partial}{\partial x} + v_y \frac{\partial}{\partial y}\right)\omega &= \nu \left(\frac{\partial^2}{\partial x^2} + \frac{\partial^2}{\partial y^2}\right)\omega \tag{2.1}
\end{align*}
\textbf{ⅱ} \(\qquad\)\text{④}式に、\(v_x = \frac{\partial \psi}{\partial y} , v_y = - \frac{\partial \psi}{\partial x}\)を代入して、

\begin{align*}
  \omega &= \frac{\partial}{\partial x}\left(-\frac{\partial \psi}{\partial x}\right) - \frac{\partial}{\partial y}\left(\frac{\partial \psi}{\partial y}\right) \\
  &= - \left(\frac{\partial^2}{\partial x^2} + \frac{\partial^2}{\partial y^2}\right)\psi \tag{2.2}
\end{align*}

\textbf{ⅲ} \(\qquad \frac{\partial}{\partial x}\text{①} + \frac{\partial}{\partial y}\text{②}\)とする。

 

\begin{gather*}
   \frac{\partial}{\partial t}\left(\frac{\partial v_x}{\partial x} + \frac{\partial v_y}{\partial y}\right) 
   + \frac{\partial}{\partial x}\left(v_x \frac{\partial}{\partial x} + v_y\frac{\partial}{\partial y}\right)v_x
   + \frac{\partial}{\partial y}\left(v_x \frac{\partial}{\partial x} + v_y\frac{\partial}{\partial y}\right) v_y \\
   = -\frac{1}{\rho}\left(\frac{\partial}{\partial x}\frac{\partial P}{\partial x} + \frac{\partial}{\partial y}\frac{\partial P}{\partial y}\right) 
   + \nu\left(\frac{\partial^2}{\partial x^2} + \frac{\partial^2}{\partial y^2}\right)\left(\frac{\partial v_x}{\partial x} + \frac{\partial v_y}{\partial y}\right)
\end{gather*}

ここで、\(\text{③}\)式より、\(\left(\frac{\partial v_x}{\partial x} + \frac{\partial v_y}{\partial y}\right) = 0\)であるため、


\begin{align*}
  -\frac{1}{\rho}\left(\frac{\partial}{\partial x}\frac{\partial P}{\partial x} + \frac{\partial}{\partial y}\frac{\partial P}{\partial y}\right)
  &=   \frac{\partial}{\partial x}\left(v_x \frac{\partial}{\partial x} + v_y\frac{\partial}{\partial y}\right)v_x
  + \frac{\partial}{\partial y}\left(v_x \frac{\partial}{\partial x} + v_y\frac{\partial}{\partial y}\right) v_y \\
  -\frac{1}{\rho}\left(\frac{\partial^2}{\partial x^2} + \frac{\partial^2}{\partial y^2}\right)P 
&= 2\left[\left(\frac{\partial^2 \psi}{\partial x\partial y}\right)^2 - \frac{\partial^2 \psi}{\partial x^2}\frac{\partial^2 \psi}{\partial y^2}\right] \\
  \therefore \qquad\left(\frac{\partial^2}{\partial x^2} + \frac{\partial^2}{\partial y^2}\right)P 
  &=  2\rho\left[\frac{\partial^2 \psi}{\partial x^2}\frac{\partial^2 \psi}{\partial y^2} - \left(\frac{\partial^2 \psi}{\partial x\partial y}\right)^2\right] \qquad \tag{2.3}
\end{align*}

以上式(2.1)~式(2.3)から、\(v_x = \frac{\partial \psi}{\partial y}, v_y = -\frac{\partial \psi}{\partial x}\)と置いたとき、式(2)が導かれた。

\end{document}
